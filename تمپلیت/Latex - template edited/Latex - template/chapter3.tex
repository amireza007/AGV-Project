\chapter{روش پژوهش}


\section{مقدمه}
(در مقدمه این فصل توضیحات کلی در مورد محتوای فصل ارائه می‌شود.)
\section{طرح، روش و یا رویکرد پژوهش}
(در این بخش با توجه به اهداف پژوهش، طرح، رویکرد، و یا روش پژوهش مشخص می‌شود. به‌طور کلی پژوهش‌ها از نظر هدف به سه روش بنیادی، کاربردی و توسعه‌ای انجام می‌شوند و از نظر گردآوری داده‎‌ها به روشهای توصیفی، همبستگی، تجربی، پیمایشی، تحلیل محتوا، تاریخی و جز آن تقسیم می‌شوند)
\section{جامعه پژوهش}
(در این بخش جامعه آماری پژوهش به تفکیک مشخصاتی مانند سن، جنس و ... توصیف می‌شود. منظور از جامعه پژوهش مجموعه  اعضای حقیقی یا فرضی مورد مطالعه است که نتیجه‌های پژوهش به آنها مربوط می‌شود.)
\section{روش نمونه‌گیری}
(در این بخش نحوه انتخاب نمونه‌های آماری، متناسب با اندازه جامعه پژوهش و روش پژوهش توصیف می‌شود. به‌طور کلی برای نمونه‌گیری از روشهای احتمالی (مانند تصادفی، نظام‌مند، طبقه‌ای، خوشه‌ای و جزء آن)  و غیراحتمالی (مانند داوطلبانه، سهمیه‌ای، هدفمند و جزء ان) استفاده می‌شود.)
\section{روش گردآوری داده‌ها}
(در این بخش با توجه به روش پژوهش، روش گردآوری داده‌ها تبیین می‌شود به‌طور کلی برای جمع‌آوری داده‌ها یکی از روش‌های مصاحبه، مشاهده، پرسشنامه و تحلیل اسناد و مدارک استفاده می‌شود.)
\section{ابزار پژوهش(گردآوری داده‌ها)}
(در این بخش ابزار گردآوری داده‌ها مانند پرسشنامه‌، سیاهه‌وارسی، فیش و جز آن توصیف می‌شود. در صورت نیاز در این بخش روش‌های اعتبارسنجی ابزار پژوهش همانند روایی و پایایی بیان می‌شود.)
\section{روش تجزیه و تحلیل داده‌ها}
(تجزیه و تحلیل داده‌ها فرایندی است که به استخراج اطلاعات ارزشمند از داده‌ها با استفاده از روش‌های کمی و کیفی اشاره دارد برای مثال در پژوهش‌های کمی آزمون‌های آماری مورد استفاده برای تجزیه و تحلیل داده‌ها ارائه می‌شود.) 
\section{ملاحظات اخلاقی}
(در این بخش از پایان‌نامه، باید محدودیت‌ها و ملاحظات اخلاقی انجام پژوهش بیان شوند. همچنین باید درمورد رضایت آگاهانه افراد برای ورود به پژوهش نیز توضیح داده شود. شایان ذکر است ملاحظات اخلاقی در پژوهش به مجموعه ای از بایدها و نبایدهای اخلاقی گفته‌ می‌شود که پژوهشگر باید در تمام مراحل پژوهش یعنی قبل از شروع به انجام پژوهش، در طول انجام پژوهش و بعد از اتمام پژوهش بدان پایبند باشد.)

