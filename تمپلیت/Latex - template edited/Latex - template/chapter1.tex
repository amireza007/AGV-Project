\chapter{معرفی پژوهش}

\section{مقدمه}
(در این بخش پژوهشگر به ارائه اطلاعات دقیق و روشن درباره موضوع مورد پژوهش، ارائه دلایل منطقی و نظری منجر به انجام پژوهش و بیان ارتباط بین این پژوهش با پژوهش‌های قبلی می‌پردازد) 
\section{بیان مسئله}
(در این بخش پژوهشگر به بیان ابهام‌ها، چالش‌ها، شکاف‌های دانشی، تعارض بین داده‌های پیشین، موارد مجهول و نیازهای موجود در رابطه با موضوع پژوهش می‌پردازد.)
\section{اهمیت و ضرورت پژوهش}
(در این بخش پژوهشگر باید درباره اهمیت، مزایا و اولویت‌های انجام پژوهش توضیح داده و ضرورت و تبعات ناشی از عدم انجام این پژوهش را بیان کند.)
\section{هدف پژوهش}
(اهداف پژوهش بیان نظام‌مند مواردی است که پژوهشگر با توجه به مسئله پژوهش به دنبال دستیابی به آن است. اهداف پژوهش شامل دو نوع اصلی و فرعی است)
\subsection{هدف اصلی}
(هدف اصلی باید به صورت یک جمله و نه عبارت بیان شود، به تمامی جنبه‌ها، متغیرها و زاویه پژوهش اشاره نماید و کاملاً همخوان با عنوان پژوهش باشد)
\subsection{هدف(های) فرعی}
(هدف اصلی پژوهش ممکن است به چند هدف فرعی تقسیم شود)
 \section{پرسش(‌ها)‌ی پژوهش}
(پرسش‌های پژوهش باید متناسب و کاملاٌ همخوان با اهداف تنظیم شوند.)
\subsection{پرسش اصلی پژوهش}
(پرسش اصلی پژوهش باید کاملا همخوان با هدف اصلی پژوهش باشد
\subsection{پرسش‌(های) فرعی پژوهش}
(پرسش اصلی پژوهش ممکن است به چند پرسش فرعی تقسیم شود و باید کاملا همخوان با اهداف فرعی پژوهش باشد)
\section{فرضیه‌ها}
(متناسب با موضوع پژوهش ممکن است به جای پرسش از فرضیه استفاده شود.  فرضیه یک حدس منطقی برای یک پدیده یا رخداد است که می‌تواند مورد آزمون قرار گیرد. تمام ویژگی‌های پرسش برای فرضیه پژوهش هم مطرح است با این تفاوت که به صورت خبری نوشته می‌شود.)
\section{متغیرهای پژوهش}
(به ویژگی‌هایی از موضوع پژوهش گفته می‌شود که پژوهشگر آنها را  اندازه‌گیری یا مشاهده می‌کند. لازم است پژوهشگر نوع متغیرهای پژوهش (مستقل، وابسته، مداخله‌گر و موارد دیگر) را مشخص کند.)
\section{تعاریف مفهومی و عملیاتی اصطلاحات}
(در تعریف مفهومی، یک اصطلاح تخصصی با استناد به منابع معتبر تعریف می‌شود. در تعریف عملیاتی، ویژگی‌ها و شاخص‌های مورد استفاده در این پژوهش بیان می‌شود.)