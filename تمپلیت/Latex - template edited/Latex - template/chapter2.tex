\chapter{مبانی نظری و پیشینه پژوهش} 

\section{مقدمه}
(در مقدمه این فصل توضیحات کلی در مورد محتوای فصل ارائه می‌شود.) 
\section{مبانی نظری}
(در این بخش پژوهشگر باید با بررسی مبانی نظری گوناگون، چهارچوب نظری عنوان پژوهش خویش را تعیین کرده و مشخص سازد. مقصود از مبانی نظری بازگویی نظریه‌ها، الگوه‌ها، چهارچوب‌ها، تعریف‌ها و رویکردهای موجود در خصوص موضوع پژوهش است.)
\section{مرور پیشینه‌های پژوهش}
(این بخش بیانگر پژوهش‌هایی است که تاکنون در رابطه با عنوان پژوهش گزارش شده اند.)
\subsection{مقدمه}
(در این بخش توضیحات مقدماتی در زمینه پیشینه‌های موضوع ارائه می‌شود که ضمن اشاره به دامنه موضوعی پژوهش‌های پیشین، به میزان جامعیت و دقت شناسایی آن پژوهش‌ها و گروه‌بندی پژوهش‌های شناسایی شده می‌پردازد.)
\subsection{پیشینه‌های داخل کشور}
(در این بخش پژوهش‌های انجام شده در داخل کشور به صورت گروه‌بندی شده مرور می‌شود. پژوهشگر باید در نگارش هر پیشینه پژوهش به نام پژوهشگر(ها)، سال انجام پژوهش، موضوع و یا هدف پژوهش، روش‌شناسی پژوهش و یافته‌های حاصل از آن بپردازد.)
\subsection{پیشینه‌های خارج از کشور}
(در این بخش پژوهش‌های انجام شده در خارج از کشور به صورت گروه‌بندی شده مرور می‌شود. پژوهشگر باید در نگارش هر پیشینه پژوهش به نام پژوهشگر(ها)، سال انجام پژوهش، موضوع و یا هدف پژوهش، روش‌شناسی پژوهش و یافته‌های حاصل از آن بپردازد.)
\subsection{نتیجه‌‌گیری}
