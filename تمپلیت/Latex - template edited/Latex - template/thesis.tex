% 
% قالب پایان نامه های دانشگاه علامه طباطبائئ (ره)


% اگر قصد نوشتن رساله دکتری را دارید، در خط زیر، گزینه msc را پاک کنید. کلیه تنظیمات لازم به طور خودکار اعمال می‌شود.
% در صورت استفاده از گزینه withpage  شماره صفحات حاوی نمادها در جلوی آن‌ها در فهرست نمادها چاپ می‌شود و در صورت 
% استفاده از گزینه printonlyused  فقط نمادهایی که در متن استفاده شده‌اند، در فهرست نمادها چاپ می‌شوند.
\documentclass[msc,withpage,printonlyused]{allameh-thesis}

\usepackage{xepersian}
\settextfont[Scale=1.3]{B Nazanin}
\defpersianfont\minutesfont[Scale=1.2]{B Nazanin}
\setlatintextfont{Times New Roman} 
\setdigitfont[Scale=1]{Times New Roman}


%\PersianMathsDigits
\SepMark{-}
\def\addsymbol #1:#2#3{$#1$ \> \parbox{5in}{#2 \dotfill \pageref{#3}}\\} 
\def\symboldisplay#1{\label{#1}} 
%For compatibility with Miktex 2.8
%\makeatletter
%\@ifundefined{Umathcode}{\let\Umathcode\XeTeXmathcode}{}
%\@ifundefined{Umathchardef}{\let\Umathchardef\XeTeXmathchardef}{}
%\makeatother 
%End Of Compatibility

\newtheorem{observation}{مشاهده}[section]

\begin{document}
\baselineskip=1cm
%\pagenumbering{alph}
%
% در صورت تمایل به تغییر لوگوی بسم‌ الله، لوگوی دلخواه خود را با اسم logo در پوشه figures قرار دهید و سپس با استفاده از دستور
% زیر پهنای آن را با عددی بین 0 و 1 مشخص کنید
%\besmwidth{.7}


% دانشکده، آموزشکده و یا پژوهشکده  خود را وارد کنید
\faculty{ دانشکده ..... }
% گروه آموزشی خود را وارد کنید
\department{گروه آموزشی ......}
% نام رشته تحصیلی خود را وارد کنید
\subject{...................}
% گرایش خود را وارد کنید
\field{......................}
% عنوان پایان‌نامه/رساله را وارد کنید
\title{...........عنوان...........}
% نام استاد(ان) راهنما را وارد کنید
\firstsupervisor{.....................}
%\secondsupervisor{استاد  دوم}
% نام استاد(دان) مشاور را وارد کنید. چنانچه استاد مشاور ندارید، دستور(های) پایین را غیرفعال کنید.
\firstadvisor{.....................}
%\secondadvisor{استاد مشاور دوم}
% نام داور را وارد کنید
\proctorname{...........}
% نام پژوهشگر را وارد کنید
\name{..........}
% نام خانوادگی پژوهشگر را وارد کنید
\surname{................}
% فصل و سال پایان‌نامه خود را وارد کنید
\thesisdate{.............}
\credit{۶}
\defensedate{.../../...}
% در صورت کامنت کردن سه دستور زیر، جای خالی به جای آن‌ها در فرم صورت‌جلسه ایجاد می‌شود
\grade{..}
\letgrade{ }
\degree{ }
% داور یا صاحبنظر داخلی اول
\firstinternalreferee{....}
% داور یا صاحبنظر داخلی دوم
\secondinternalreferee{ }
% داور یا صاحبنظر خارجی اول
\firstexternalreferee{ }
% داور یا صاحبنظر خارجی دوم
% ناظر جلسه دفاع
\viewer{ }
%%%%%%%%%%%%%%%%%%%%%%%%%%%%%%%%%%%%
\totext{%
\noindent{\Large\bfseries تقدیم به }\\
\vspace*{1em}
\begin{center}
\large\bfseries
تمامی کسانی که مرا همراهی نمودند
\end{center}
\begin{center}
و همه کسانی که درست اندیشیدن را به من آموختند.
%%%%دکتر علی شریعتی
\end{center}
}
%%%%%%%%%%%%%%%%%%%%%%%%%%%%%%%%%%%%
\ack{%
\subsection*{سپاس‌گزاری}
سپاس خداوند یکتای عزتمندی که رحمت و دانش او در سراسر گیتی گسترده شده، آسمان‌ها و زمین همه از آن اوست و  علم و دانش حقیقی را بر هر که بخواهد موهبت می‌فرماید. رحمت و لطف او را بی‌نهایت سپاس می‌گویم چرا که فهم و درک مطالب این پژوهش را بر من ارزانی داشت و مرا به این اصل رساند که علم و ایمان دو بال یک پروازند. توفیق تلاش به‌ من داد و هر بار که خطا کردم فرصتی دوباره، تا با امید، تلاشی تازه را آغاز کنم و به خواست او به نتیجه‌ی مطلوب نائل آیم. به‌راستی که همه چیز از آن اوست و همه‌چیز به خواست اوست. 
%%%% در صورت استفاده از دستور زیر، تاریخ و امضای شما، به طور خودکار درج می‌شود.
%\signature 
}
%%%%%%%%%%%%%%%%%%%%%%%%%%%%%%%%%%%%
\allamehtitle

\thispagestyle{plain}
\addcontentsline{toc}{section}{ چکیده}
\section*{چکیده}
(چکیده پایان‌نامه‌ها/ رساله‌ها از نوع تمام‌نما است. محتوای آن به ترتیب شامل هدف، روش‌شناسی، یافته‌ها، و نتیجه‌گیری است. چکیده باید در یک پاراگراف تنظیم و بین 150 تا 300 واژه باشد. محتوای چکیده  باید به گونه‌ای تنظیم شود که خواننده بدون مراجعه به متن، ایده اصلی متن را دریافت نموده و از مطالعه متن بی‌نیاز شود.)


\vfil

\section*{کلمات کلیدی:}
...،. . . . . .، . . . . .، . . . . . .
(تعداد کلید‌واژه‌ها یا عبارات کلیدی باید بین 3 تا 10 واژه یا عبارت باشد و دقیقاً از متن چکیده  استخرج شده باشند. وا‌ژه‌ها و اصطلاحات یاد شده باید به ترتیب الفبایی تنظیم  و با ویرگول از هم جدا شوند.)

\tableofcontents

\listoftables%\clearpage
\pagenumbering{alph}
\listoffigures%\clearpage
%\mychapter{فهرست نمادها}
\thispagestyle{plain}
\begin{multicols*}{2}
% پهن‌ترین نماد باید داخل کروشه زیر قرار بگیرد تا توضیحات نمادها زیر هم تراز شوند.
\begin{acronym}[\lr{CPU}]
\baselineskip=.7cm
\acro{real}[$\mathbb{R}$]{مجموعه اعداد حقیقی}
\acro{im}[$\mathbb{C}$]{مجموعه اعداد موهومی}
\acro{nat}[$\mathbb{N}$]{مجموعه اعداد طبیعی}
\acro{cpu}[\lr{CPU}]{\lr{Central Processing Unit}}
\end{acronym}
\end{multicols*}
%\begin{tabbing}
%\parbox{0.17\textwidth}{نماد}\=\parbox{0.7\textwidth}{توضیح\hfill صفحه}\\
%\addsymbol \lambda_{s}(n): {طول طولانی‌ترین $(n, s)$ دنباله‌ی داونپورت-شینزل}{landaDavShinz}
%\addsymbol \cal F: {یک مجموعه از توابع}{setOfFunc}
%\addsymbol \Gamma(\cal F): {پوشش پایینی مجموعه توابع $\cal F$}{lowerEnvelop}
%\addsymbol \alpha(n): {معکوس تابع آکرمان}{alphaN}
%\end{tabbing}

\cleardoublepage


\pagenumbering{arabic}
\setcounter{page}{1}

\chapter{معرفی پژوهش}

\section{مقدمه}
(در این بخش پژوهشگر به ارائه اطلاعات دقیق و روشن درباره موضوع مورد پژوهش، ارائه دلایل منطقی و نظری منجر به انجام پژوهش و بیان ارتباط بین این پژوهش با پژوهش‌های قبلی می‌پردازد) 
\section{بیان مسئله}
(در این بخش پژوهشگر به بیان ابهام‌ها، چالش‌ها، شکاف‌های دانشی، تعارض بین داده‌های پیشین، موارد مجهول و نیازهای موجود در رابطه با موضوع پژوهش می‌پردازد.)
\section{اهمیت و ضرورت پژوهش}
(در این بخش پژوهشگر باید درباره اهمیت، مزایا و اولویت‌های انجام پژوهش توضیح داده و ضرورت و تبعات ناشی از عدم انجام این پژوهش را بیان کند.)
\section{هدف پژوهش}
(اهداف پژوهش بیان نظام‌مند مواردی است که پژوهشگر با توجه به مسئله پژوهش به دنبال دستیابی به آن است. اهداف پژوهش شامل دو نوع اصلی و فرعی است)
\subsection{هدف اصلی}
(هدف اصلی باید به صورت یک جمله و نه عبارت بیان شود، به تمامی جنبه‌ها، متغیرها و زاویه پژوهش اشاره نماید و کاملاً همخوان با عنوان پژوهش باشد)
\subsection{هدف(های) فرعی}
(هدف اصلی پژوهش ممکن است به چند هدف فرعی تقسیم شود)
 \section{پرسش(‌ها)‌ی پژوهش}
(پرسش‌های پژوهش باید متناسب و کاملاٌ همخوان با اهداف تنظیم شوند.)
\subsection{پرسش اصلی پژوهش}
(پرسش اصلی پژوهش باید کاملا همخوان با هدف اصلی پژوهش باشد
\subsection{پرسش‌(های) فرعی پژوهش}
(پرسش اصلی پژوهش ممکن است به چند پرسش فرعی تقسیم شود و باید کاملا همخوان با اهداف فرعی پژوهش باشد)
\section{فرضیه‌ها}
(متناسب با موضوع پژوهش ممکن است به جای پرسش از فرضیه استفاده شود.  فرضیه یک حدس منطقی برای یک پدیده یا رخداد است که می‌تواند مورد آزمون قرار گیرد. تمام ویژگی‌های پرسش برای فرضیه پژوهش هم مطرح است با این تفاوت که به صورت خبری نوشته می‌شود.)
\section{متغیرهای پژوهش}
(به ویژگی‌هایی از موضوع پژوهش گفته می‌شود که پژوهشگر آنها را  اندازه‌گیری یا مشاهده می‌کند. لازم است پژوهشگر نوع متغیرهای پژوهش (مستقل، وابسته، مداخله‌گر و موارد دیگر) را مشخص کند.)
\section{تعاریف مفهومی و عملیاتی اصطلاحات}
(در تعریف مفهومی، یک اصطلاح تخصصی با استناد به منابع معتبر تعریف می‌شود. در تعریف عملیاتی، ویژگی‌ها و شاخص‌های مورد استفاده در این پژوهش بیان می‌شود.)
\chapter{مبانی نظری و پیشینه پژوهش} 

\section{مقدمه}
(در مقدمه این فصل توضیحات کلی در مورد محتوای فصل ارائه می‌شود.) 
\section{مبانی نظری}
(در این بخش پژوهشگر باید با بررسی مبانی نظری گوناگون، چهارچوب نظری عنوان پژوهش خویش را تعیین کرده و مشخص سازد. مقصود از مبانی نظری بازگویی نظریه‌ها، الگوه‌ها، چهارچوب‌ها، تعریف‌ها و رویکردهای موجود در خصوص موضوع پژوهش است.)
\section{مرور پیشینه‌های پژوهش}
(این بخش بیانگر پژوهش‌هایی است که تاکنون در رابطه با عنوان پژوهش گزارش شده اند.)
\subsection{مقدمه}
(در این بخش توضیحات مقدماتی در زمینه پیشینه‌های موضوع ارائه می‌شود که ضمن اشاره به دامنه موضوعی پژوهش‌های پیشین، به میزان جامعیت و دقت شناسایی آن پژوهش‌ها و گروه‌بندی پژوهش‌های شناسایی شده می‌پردازد.)
\subsection{پیشینه‌های داخل کشور}
(در این بخش پژوهش‌های انجام شده در داخل کشور به صورت گروه‌بندی شده مرور می‌شود. پژوهشگر باید در نگارش هر پیشینه پژوهش به نام پژوهشگر(ها)، سال انجام پژوهش، موضوع و یا هدف پژوهش، روش‌شناسی پژوهش و یافته‌های حاصل از آن بپردازد.)
\subsection{پیشینه‌های خارج از کشور}
(در این بخش پژوهش‌های انجام شده در خارج از کشور به صورت گروه‌بندی شده مرور می‌شود. پژوهشگر باید در نگارش هر پیشینه پژوهش به نام پژوهشگر(ها)، سال انجام پژوهش، موضوع و یا هدف پژوهش، روش‌شناسی پژوهش و یافته‌های حاصل از آن بپردازد.)
\subsection{نتیجه‌‌گیری}

\chapter{روش پژوهش}


\section{مقدمه}
(در مقدمه این فصل توضیحات کلی در مورد محتوای فصل ارائه می‌شود.)
\section{طرح، روش و یا رویکرد پژوهش}
(در این بخش با توجه به اهداف پژوهش، طرح، رویکرد، و یا روش پژوهش مشخص می‌شود. به‌طور کلی پژوهش‌ها از نظر هدف به سه روش بنیادی، کاربردی و توسعه‌ای انجام می‌شوند و از نظر گردآوری داده‎‌ها به روشهای توصیفی، همبستگی، تجربی، پیمایشی، تحلیل محتوا، تاریخی و جز آن تقسیم می‌شوند)
\section{جامعه پژوهش}
(در این بخش جامعه آماری پژوهش به تفکیک مشخصاتی مانند سن، جنس و ... توصیف می‌شود. منظور از جامعه پژوهش مجموعه  اعضای حقیقی یا فرضی مورد مطالعه است که نتیجه‌های پژوهش به آنها مربوط می‌شود.)
\section{روش نمونه‌گیری}
(در این بخش نحوه انتخاب نمونه‌های آماری، متناسب با اندازه جامعه پژوهش و روش پژوهش توصیف می‌شود. به‌طور کلی برای نمونه‌گیری از روشهای احتمالی (مانند تصادفی، نظام‌مند، طبقه‌ای، خوشه‌ای و جزء آن)  و غیراحتمالی (مانند داوطلبانه، سهمیه‌ای، هدفمند و جزء ان) استفاده می‌شود.)
\section{روش گردآوری داده‌ها}
(در این بخش با توجه به روش پژوهش، روش گردآوری داده‌ها تبیین می‌شود به‌طور کلی برای جمع‌آوری داده‌ها یکی از روش‌های مصاحبه، مشاهده، پرسشنامه و تحلیل اسناد و مدارک استفاده می‌شود.)
\section{ابزار پژوهش(گردآوری داده‌ها)}
(در این بخش ابزار گردآوری داده‌ها مانند پرسشنامه‌، سیاهه‌وارسی، فیش و جز آن توصیف می‌شود. در صورت نیاز در این بخش روش‌های اعتبارسنجی ابزار پژوهش همانند روایی و پایایی بیان می‌شود.)
\section{روش تجزیه و تحلیل داده‌ها}
(تجزیه و تحلیل داده‌ها فرایندی است که به استخراج اطلاعات ارزشمند از داده‌ها با استفاده از روش‌های کمی و کیفی اشاره دارد برای مثال در پژوهش‌های کمی آزمون‌های آماری مورد استفاده برای تجزیه و تحلیل داده‌ها ارائه می‌شود.) 
\section{ملاحظات اخلاقی}
(در این بخش از پایان‌نامه، باید محدودیت‌ها و ملاحظات اخلاقی انجام پژوهش بیان شوند. همچنین باید درمورد رضایت آگاهانه افراد برای ورود به پژوهش نیز توضیح داده شود. شایان ذکر است ملاحظات اخلاقی در پژوهش به مجموعه ای از بایدها و نبایدهای اخلاقی گفته‌ می‌شود که پژوهشگر باید در تمام مراحل پژوهش یعنی قبل از شروع به انجام پژوهش، در طول انجام پژوهش و بعد از اتمام پژوهش بدان پایبند باشد.)


\chapter{تجزیه و تحلیل داده‌ها}

\section{مقدمه}
(در مقدمه این فصل توضیحات کلی در مورد محتوای فصل ارائه می‌شود.)
\section{تجزیه و تحلیل داده‌ها}
(در این بخش، داده‌های گردآوری شده در فرآیند اجرای پژوهش در قالب متن، آزمون آماری (توصیفی و استنباطی) ، نمودار، و جدول بر مبنای پرسش‌ها یا فرضیه‌های پژوهش گروه‌بندی شده و سپس تجزیه و تحلیل می‌شوند.)
\subsection{پاسخ به پرسش‌اول پژوهش}
\subsection{پاسخ به پرسش‌ دوم پژوهش}
\subsection{پاسخ به پرسش‌ سوم پژوهش}




\chapter{بحث و نتیجه‏ گیری}

\section{مقدمه}
(در مقدمه این فصل توضیحات کلی در مورد محتوای فصل ارائه می‌شود.)
\section{بحث و نتیجه‌گیری}
(در این بخش داده‌های گردآوری شده در فرآیند اجرای پژوهش که در فصل چهارم ارائه شده بودند، مورد تفسیر و استنباط قرار گرفته، یافته‌های پژوهش تبیین و همخوان با پرسش‌ها و یا فرضیه‌های پژوهش بیان می‌شود.)
\subsection{بحث و نتیجه‌گیری درباره پرسش اول پژوهش}
\subsection{بحث و نتیجه‌گیری درباره پرسش دوم پژوهش}
\subsection{بحث و نتیجه‌گیری درباره فرضیه پژوهش}
\section{مقایسه یافته‌های پژوهش با یافته‌های پژوهش‌های پیشین}
\section{جمع‌بندی}
\section{محدودیت‌های پژوهش}
\section{پیشنهاد‌هایی برای پژوهش‌های آتی}
\section{پیشنهاد‌های اجرایی پژوهش}



\appendix
\chapter{پیوست}

\mychapter{واژه‌نامه فارسی به انگلیسی}

\englishgloss{Probabilistic}{احتمالی}
\englishgloss{Measure}{اندازه }
\englishgloss{Stably}{پایدار}
\englishgloss{Weak Topology}{توپولوژی ضعیف}
\englishgloss{Powerdomain}{دامنه‌توانی}
\englishgloss{Function Space}{فضای تابع}
\englishgloss{Semantic Domain}{دامنه معنایی}
\englishgloss{Program Fragment}{قطعه‌برنامه}
\englishgloss{Ordered}{مرتب}
\englishgloss{Dcpo}{مجموعه جزئاً مرتب کامل جهت‌دار}
\englishgloss{Valuation}{ارزیابی}

\cleardoublepage
\mychapter{واژه‌نامه  انگلیسی به  فارسی}


\persiangloss{مجموعه جزئاً مرتب کامل جهت‌دار}{Dcpo}
\persiangloss{فضای تابع}{Function Space}
\persiangloss{اندازه }{Measure}
\persiangloss{مرتب}{Ordered}
\persiangloss{دامنه‌توانی}{Powerdomain}
\persiangloss{احتمالی}{Probabilistic}
\persiangloss{قطعه‌برنامه}{Program Fragment}
\persiangloss{دامنه معنایی}{Semantic Domain}
\persiangloss{پایدار}{Stably}
\persiangloss{ارزیابی}{Valuation}
\persiangloss{توپولوژی ضعیف}{Weak Topology}

\bibliographystyle{acm-fa}
\bibliography{MyReferences}
\printindex
% در این فایل، عنوان پایان‌نامه، مشخصات خود و چکیده پایان‌نامه را به انگلیسی وارد کنید.
%%%%%%%%%%%%%%%%%%%%%%%%%%%%%%%%%%%%
\begin{latin}
%عنوان دانشکده 
\latinfaculty{.......}
%عنوان گروه 
\latindepartment{.......... }
%عنوان لاتین پایان نامه
\latintitle{.....................}
% استاد راهنما
\firstlatinsupervisor{................}
% استاد مشاور
\firstlatinadvisor{...............}
% نام  
\latinname{...............}
% نام خانوادگی
\latinsurname{.................}
% ماه و سال میلادی دفاع را وارد کنید
\latinthesisdate{.........}
\enabstract{
\subsection*{Abstract}





















\subsection*{Keywords}
(3 to 10 keywords that sorted by Alphabet)
\thispagestyle{empty}

% چکیده

}
\latinallamehtitle
\end{latin}

\end{document}




%%%%%%%%%%
%گردآوری مهدی شفیعی
% shafiee@atu.ac.ir
%بهار 96
%%%